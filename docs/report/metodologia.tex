\section*{Enfoque Metodológico}

Dado que el proyecto que se desarrolla tiene un tiempo de desarrollo definido y unas especificaciones inmutables, la adopción de algunas de las metodologías de desarrollo de software ágiles en realidad solo aportaría una pequeña parte de los beneficios de dichas técnicas, mientras que sus desventajas no se atenuarían.

Por esto se decidió la adopción de una metodología híbrida que combina la planificación por adelantado de \textbf{RUP} con el desarrollo orientado a tests y el desarrollo en parejas de la programación extrema (\textbf{XP}). En concreto, lo que se pretende es un desarrollo iterativo horizontal típico de las técnicas ágiles usando test unitarios planeados de antemano para verificar cada iteración; al final de cada iteración, como el proyecto no plantea entregas periódicas de software (ni cambios en los requerimientos), simplemente se continuará con la siguiente.

Para concretar, la metodología adoptada tiene como características:

\begin{enumerate}
  \item[\(\cdot\)] Planificación previa de la fases de desarrollo y de las características que se añadirán al producto final.
  \item[\(\cdot\)] Construcción previa de tests unitarios usando para ello los casos de uso identificados durante el proceso de planificación.
  \item[\(\cdot\)] Desarrollo iterativo horizontal en la que cada iteración implementa una característica (posiblemente un caso de uso) dada.
  \item[\(\cdot\)] Revisión constante de código gracias a la implementación del desarrollo en parejas.
\end{enumerate}
