% Copyright (c) 2023-2025
% This file is part of sep3cs.
%
% sep3cs is free software: you can redistribute it and/or modify
% it under the terms of the GNU General Public License as published by
% the Free Software Foundation, either version 3 of the License, or
% (at your option) any later version.
%
% sep3cs is distributed in the hope that it will be useful,
% but WITHOUT ANY WARRANTY; without even the implied warranty of
% MERCHANTABILITY or FITNESS FOR A PARTICULAR PURPOSE.  See the
% GNU General Public License for more details.
%
% You should have received a copy of the GNU General Public License
% along with sep3cs. If not, see <http://www.gnu.org/licenses/>.
%
\section{Introducción}

\subsection{Alcance del producto}

El sistema a desarrollar es una herramienta auxiliar del popular juego de móvil Clash Royale. Su principal función es la de recopilar y almacenar datos de interés del juego antes mencionado, ya sea sobre jugadores, cartas, clanes, desafíos u otros elementos del juego; y poner estos a disposición de los usuarios, también se encarga de analizar y organizar estos datos datos de forma que resulte más simple obtener la información deseada. Esto no solo sirve como orientación para nuevos jugadores que pueden acceder fácilmente a datos como tales como las cartas más populares o los mejores desafíos sino que permite a los clanes y jugadores veteranos, a través del registro de sus logros en el juego, fomentar la competitividad y el sentido de logro, lo que mejora su experiencia de juego y aumenta el atractivo del mismo para nuevos usuarios.

\subsection{Descripción general del producto}

Nuestro sistema está destinado a mejorar el atractivo del juego Clash Royale por medio de un fácil acceso a la información del mismo lo que , como habíamos mencionado anteriormente, funciona como una especie de guía para novatos y un incentivo para todos los jugadores, motivándolos a querer alcanzar logros que sean registrados y estén a la vista de toda la comunidad. Esto, esperamos, aumente el interés en  el juego, y por tanto el número de jugadores lo que a su vez aumentará el número de usuarios de nuestro sistema. 

Las funciones principales de nuestro producto se centran en recopilar, analizar y poner a disposición del usuario datos referentes a las características de distintos elementos del juego, estas son:

\begin{enumerate}
  \item[\(\cdot\)] Sobre los jugadores: Código ID, apodo, nivel, cantidad de victorias, carta favorita actual, cantidad de trofeos, máximo de trofeos alcanzados, cantidad de cartas encontradas, las cartas que posee, en que nivel está cada carta, clan al que pertenece, posición que ocupa en el clan, las cartas que ha donado a un clan, los desafíos en los que ha participado, los premios que ha alcanzado en cada desafío, las partidas que ha jugado, su resultado en cada partida.
  \item[\(\cdot\)] Sobre los desafíos: ID desafío, nombre, descripción, costo, nivel mínimo para participar, cantidad de derrotas que admite, tiempo de duración, fecha de comienzo, cantidad de premios que ofrece.
  \item[\(\cdot\)] Sobre los clanes: Código ID Clan, nombre, descripción, tipo, región, cantidad de miembros, cantidad de trofeos necesarios para entrar, cantidad de trofeos obtenidos en guerras.
  \item[\(\cdot\)] Sobre cualquier carta en general: ID Carta, nombre, descripción, nivel inicial, costo de elixir ,calidad.
  \item[\(\cdot\)] Sobre las cartas de tipo tropa: Puntos de vida, daño en área, cantidad de unidades.
  \item[\(\cdot\)] Sobre las cartas de tipo estructura: Puntos de vida, velocidad de ataque, daño a distancia.
  \item[\(\cdot\)] Sobre las cartas de tipo hechizo: Radio que afecta, duración, daño en área, daño en torres enemigas.
  \item[\(\cdot\)] Sobre las guerras de clanes: ID Guerra, fecha de comienzo.
  \item[\(\cdot\)] Sobre las partidas: ID Partida, fecha y hora de comienzo, duración, ganador.
\end{enumerate}

El sistema luego debe usar estos datos para responder las consultas de los usuarios, estando disponibles para ellos en todo momento. Los datos tienen que actualizarse constantemente. 

Como un sistema que sirve como complemento al juego Clash Royale, los usuarios de nuestro producto serán precisamente los jugadores de este, como tal, aunque siempre hay excepciones, no se espera que posean un conocimiento técnico sobre programación o manejo de datos, por lo que la interfaz de usuario debe mantener esto en cuenta y ser lo más simple e intuitiva posible. 

Nuestro sistema también se ve afectado por restricciones como son:

\begin{enumerate}
  \item[\(\cdot\)] Las políticas y regulaciones relacionadas con la privacidad y seguridad de los datos. Si bien hay algunos datos, como la cantidad de trofeos obtenidos en guerras por un clan, que puede resultar de interés para ellos difundirlos a fin de aumentar su popularidad y atraer nuevos miembros, hay datos confidenciales cuya libre difusión no solo viola la privacidad de los usuarios sino que supondría una desventaja para ellos en el juego, como las cartas poseídas por un jugador y su nivel.
  \item[\(\cdot\)] La necesidad de compatibilidad del sistema con las plataformas en que se juega Clash Royale, lo cual incluía anteriormente solo a iOS y Android, pero recientemente se está produciendo una versión del mismo para PC.
  \item[\(\cdot\)] Garantizar un nivel bajo de requisitos de conocimiento técnico para sus uso, en otras palabras, que sea simple e intuitivo de forma que esté disponible para la mayor cantidad de usuarios. 
\end{enumerate}
